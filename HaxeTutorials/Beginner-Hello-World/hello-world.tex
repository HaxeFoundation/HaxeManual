\article{Hello World}
\label{hello-world}
\maintainer{Simon Krajewski}

\subtoc

\section*{Introduction}
\label{hello-world-introduction}

This tutorial demonstrates how to write and compile a Hello World Haxe program. It explains the involved file-formats (.hx and .hxml files) and gives a basic explanation of what the Haxe Compiler does with them.

\paragraph{Requirements}

\begin{itemize}
	\item Haxe has to be installed and available from command line.
	\item You have to know how to to save files on your computer.
	\item You have to be able to open a command line, navigate to a directory and execute a command.
\end{itemize}

\section*{Creating and saving the code}
\label{hello-world-code}

Copy and paste the following code into any editor or IDE of your choice:

\haxe{../HaxeManual/assets/HelloWorld.hx}

Save it as ``HelloWorld.hx'' anywhere you like.

\section*{Executing Haxe to interpret the code}
\label{hello-world-executing}

Open a command prompt and navigate directories to where you saved ``HelloWorld.hx'' to in the \tref{previous step}{hello-world-code}. Afterwards, execute this command:

\begin{lstlisting}
haxe -main HelloWorld --interp
\end{lstlisting}