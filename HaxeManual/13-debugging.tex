\chapter{Debugging}
\label{debugging}
\state{NoContent}

\section{Logging and Trace}
\label{debugging-trace-log}

Haxe provides developers with a powerful logging/trace system. Simply call \expr{trace} within functions:

\begin{lstlisting}
trace("Hello world!");
\end{lstlisting}

In most Haxe targets trace will be printed to stdout. JavaScript uses \ic{console.log}. Each trace is displayed with the filename and linenumber information where the trace occured:

\begin{lstlisting}
Test.hx:11: Hello world!
\end{lstlisting}

To trace without the default position information \ic{haxe.Log.trace(msg, null)} can be used.

\paragraph{Custom trace}

The trace can have a custom output by changing the \expr{Log.trace} method where all trace calls are redirected. 

\haxe{assets/CustomTrace.hx}

The \ic{v} argument is the first parameter of the trace call. It can be a \expr{String} or any other value. The optional \ic{infos} argument contains extra position parameter, see below.

The \expr{infos.customParams} array contains all extra arguments that were given to the original trace. If no extra parameters are passed, it will be \expr{null}. 

As illustration, the previous example will be compiled as if it was calling the following:

\begin{lstlisting}
haxe.Log.trace("hello", {
	fileName : "Test.hx", 
	lineNumber : 6, 
	className : "Test", 
	methodName : "main", 
	customParams : ["warning",123]
});
\end{lstlisting}

\paragraph{Removing traces}

You can simply remove all trace informations by compiling your project with \ic{--no-traces} argument. This will remove all trace calls as if they were not present in the program.

\section{Position Information Parameter}
\label{debugging-posinfos}

\href{http://api.haxe.org/haxe/PosInfos.html}{haxe.PosInfos} is a magic type which can be used to generate position information into the output for debugging use.
If a function has a final optional argument of this type, i.e. \expr{(..., ?pos:haxe.PosInfos)}, each call to that function which does not assign a value to that argument has its position added as call argument. 

It is sometimes useful to define a custom method that does some traces in some case. The following usage is possible since in Haxe when the \expr{haxe.PosInfos} optional parameter is not set, its default value will always be replaced by the compiler:

\haxe{assets/AssertTrace.hx}

\section{Tracing types}
\label{debugging-type-function}

\expr{\$type} is a \emph{compile-time} mechanism being called like a function, with a single argument. The compiler evaluates the argument expression and then outputs the type of that expression.

This is useful to evaluate if an expression has a certain type, mostly when dealing with \tref{Type inference}{type-system-type-inference}, which leaves the definition of the type up to the compiler.

\begin{lstlisting}
var myValue = "foo";
$type(myValue); // String
\end{lstlisting}

