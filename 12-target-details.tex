\chapter{Target Details}
\label{target-details}

\section{Javascript}
\label{target-javascript}
\state{NoContent}

\subsection{Loading extern classes using "require" function}
\label{target-javascript-require}
\since{3.2.0}

Modern \target{JavaScript} platforms, such as Node.js provide a way of loading objects
from external modules using the "require" function. Haxe supports automatic generation
of "require" statements for \expr{extern} classes.

This feature can be enabled by specifying \expr{@:jsRequire} metadata for the extern class. If our \expr{extern} class represents a whole module, we pass a single argument to the \expr{@:jsRequire} metadata specifying the name of the module to load:

\haxe{assets/JSRequireModule.hx}

In case our \expr{extern} class represents an object within a module, second \expr{@:jsRequire} argument specifies an object to load from a module:

\haxe{assets/JSRequireObject.hx}

The second argument is a dotted-path, so we can load sub-objects in any hierarchy.

If we need to load custom JavaScript objects in runtime, a \expr{js.Lib.require} function can be used. It takes \expr{String} as its only argument and returns \expr{Dynamic}, so it is advised to be assigned to a strictly typed variable.

\section{Flash}
\label{target-flash}

\subsection{Getting started with Flash}
\label{target-flash-getting-started}

Developing Flash applications is really easy with Haxe. Let's see our first sample.
This is a very simple example showing most of the toolchain. 

Create a new folder, save this class as \ic{Main.hx}.

\begin{lstlisting}
import flash.Lib;
import flash.display.Shape;
class Main {
    static function main() {
        var stage = Lib.current.stage;
        
        // create a center aligned rounded gray square
        var shape = new Shape();
        shape.graphics.beginFill(0x333333);
		shape.graphics.drawRoundRect(0, 0, 100, 100, 10);
		shape.x = (stage.stageWidth - 100) / 2;
		shape.y = (stage.stageHeight - 100) / 2;
		
		stage.addChild(shape);
    }    
}
\end{lstlisting}

To compile, you can run this from the command line:

\begin{lstlisting}
haxe -swf main-flash.swf -main Main -swf-version 15 -swf-header 960:640:60:f68712
\end{lstlisting}

.. or create and run (double-click) a file called \ic{compile.hxml}. In this example the hxml-file should be in the same directory as the example class.

\begin{lstlisting}
-swf main-flash.swf
-main Main
-swf-version 15
-swf-header 960:640:60:f68712
\end{lstlisting}

The output will be a main-flash.swf with size 960x640 pixels at 60 FPS with a orange background color and a gray square in the center.

\paragraph{Display the Flash}

You can run the SWF standalone using the \href{https://www.adobe.com/support/flashplayer/downloads.html}{Standalone Debugger FlashPlayer}. 

To display the output in a browser using the Flash-plugin, you need to create a html-document called \ic{index.html} and open it.

\begin{lstlisting}
<!DOCTYPE html>
<html>
	<body>
		<embed src="main-flash.swf" width="960" height="640">
	</body>
</html>
\end{lstlisting}

\paragraph{Differences}

The \tref{standard classes}{std} such as \type{Date}, \type{Array}, \type{Xml} or \type{Http} can have some changes in Haxe since they are common to all Haxe platforms.

Apart from these minor differences, you can easily use all of the existing Flash API if you know it already. 
You can read the more about the Actionscript 3 documentation in the \href{http://help.adobe.com/en_US/FlashPlatform/reference/actionscript/3/}{Adobe Livedocs}.

\subsection{Embedding resources}
\label{target-flash-resources}

Embedding resources is different in Haxe compared to Actionscript 3. Instead of using \ic{\[embed\]} (AS3-metatag) you'll have to use \tref{Flash specific compiler metatags}{target-flash-metatags} like \ic{@:bitmap}, \ic{@:font}, \ic{@:sound} or \ic{@:file}.

\begin{lstlisting}
import flash.Lib;
import flash.display.BitmapData;
import flash.display.Bitmap;

class Main {
  public static function main() {
    var img = new Bitmap( new MyBitmapData(0, 0) );
    Lib.current.addChild(img);
  }
}

@:bitmap("relative/path/to/myfile.png") 
class MyBitmapData extends BitmapData { }
\end{lstlisting}

\subsection{Using external Flash libraries}
\label{target-flash-external-libraries}

To embed external \ic{.swf} or \ic{.swc} libraries, you have the following \href{http://haxe.org/documentation/introduction/compiler-usage.html}{compilation options}:

\begin{description}
	\item[\expr{-swf-lib <file>}] Embeds the SWF library in the compiled SWF.
	\item[\expr{-swf-lib-extern <file>}] Add the SWF library for type checking but don't include it in the compiled SWF.
\end{description}

The standard compilation options also provides you more Haxe sources to be added to your project:

\begin{itemize}
	\item To add another class path you can use \expr{-cp <directory>}.
	\item To add a \tref{Haxelib library}{haxelib} you can use \expr{-lib <library-name>}.
	\item To force a whole package to be included in the project, use \expr{--macro include('mypackage')} that will include all the classes declared in the given package and subpackages. 
\end{itemize}

\subsection{Using Flash Metatags}
\label{target-flash-metatags}

This is the list of Flash specific metatags. See also the complete list of all \tref{Haxe built-in metatags}{cr-metadata}.

\begin{center}
\begin{tabular}{| l | l | l |}
	\hline
	\multicolumn{3}{|c|}{Flash metatags} \\ \hline
	Metatag &  Description  &  Platform \\ \hline
	@:bind  &  Override Swf class declaration  &  flash \\
	@:bitmap \_(Bitmap file path)\_  &  \_Embeds given bitmap data into the class (must extend \expr{flash.display.BitmapData})   &  flash \\
	@:debug  &  Forces debug information to be generated into the Swf even without \expr{-debug}   &  flash \\
	@:file(File path)  &  Includes a given binary file into the target Swf and associates it with the class (must extend \expr{flash.utils.ByteArray})  &  flash \\
	@:font \_(TTF path Range String)\_  &  Embeds the given TrueType font into the class (must extend \expr{flash.text.Font})  &  flash \\
	@:getter \_(Class field name)\_  &  Generates a native getter function on the given field   &  flash \\
	@:noDebug &  Does not generate debug information into the Swf even if \expr{-debug} is set   &  flash \\
	@:ns  &  Internally used by the Swf generator to handle namespaces   &  flash \\
	@:setter \_(Class field name)\_  &  Generates a native getter function on the given field   &  flash \\
	@:sound \_(File path)\_  &  Includes a given \_.wav\_ or \_.mp3\_ file into the target Swf and associates it with the class (must extend \expr{flash.media.Sound})  &  flash \\
\end{tabular}
\end{center}

\section{Neko}
\label{target-neko}

\section{PHP}
\label{target-php}

\section{C++}
\label{target-cpp}

\subsection{Using C++ Defines}
\label{target-cpp-defines}
\begin{itemize}
	\item ANDROID_HOST
	\item ANDROID_NDK_DIR
	\item ANDROID_NDK_ROOT
	\item BINDIR
	\item DEVELOPER_DIR
	\item HXCPP
	\item HXCPP_32
	\item HXCPP_COMPILE_CACHE
	\item HXCPP_COMPILE_THREADS
	\item HXCPP_CONFIG
	\item HXCPP_CYGWIN
	\item HXCPP_DEPENDS_OK
	\item HXCPP_EXIT_ON_ERROR
	\item HXCPP_FORCE_PDB_SERVER
	\item HXCPP_M32
	\item HXCPP_M64
	\item HXCPP_MINGW
	\item HXCPP_MSVC
	\item HXCPP_MSVC_CUSTOM
	\item HXCPP_MSVC_VER
	\item HXCPP_NO_COLOR
	\item HXCPP_NO_COLOUR
	\item HXCPP_VERBOSE
	\item HXCPP_WINXP_COMPAT
	\item IPHONE_VER
	\item LEGACY_MACOSX_SDK
	\item LEGACY_XCODE_LOCATION
	\item MACOSX_VER
	\item MSVC_VER
	\item NDKV
	\item NO_AUTO_MSVC
	\item PLATFORM
	\item QNX_HOST
	\item QNX_TARGET
	\item TOOLCHAIN_VERSION
	\item USE_GCC_FILETYPES
	\item USE_PRECOMPILED_HEADERS
	\item android
	\item apple
	\item blackberry
	\item cygwin
	\item dll_import
	\item dll_import_include
	\item dll_import_link
	\item emcc
	\item emscripten
	\item gph
	\item hardfp
	\item haxe_ver
	\item ios
	\item iphone
	\item iphoneos
	\item iphonesim
	\item linux
	\item linux_host
	\item mac_host
	\item macos
	\item mingw
	\item rpi
	\item simulator
	\item tizen
	\item toolchain
	\item webos
	\item windows
	\item windows_host
	\item winrt
	\item xcompile
\end{itemize}

\subsection{Using C++ Pointers}
\label{target-cpp-pointers}

\section{Java}
\label{target-java}

\section{C\#}
\label{target-cs}
