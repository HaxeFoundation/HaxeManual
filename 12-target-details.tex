\chapter{Target Details}
\label{target-details}

\section{Javascript}
\label{target-javascript}

\subsection{Loading extern classes using "require" function}
\label{target-javascript-require}

Modern \target{JavaScript} platforms, such as Node.js provide a way of loading objects
from external modules using the "require" function. Haxe supports automatic generation
of "require" statements for \expr{extern} classes.

This feature can be enabled by specifying \expr{@:jsRequire} metadata for the extern class. If our \expr{extern} class represents a whole module, we pass a single argument to the \expr{@:jsRequire} metadata specifying the name of the module to load:

\haxe{assets/JSRequireModule.hx}

In case our \expr{extern} class represents an object within a module, second \expr{@:jsRequire} argument specifies an object to load from a module:

\haxe{assets/JSRequireObject.hx}

The second argument is a dotted-path, so we can load sub-objects in any hierarchy.

If we need to load custom JavaScript objects in runtime, a \expr{js.Lib.require} function can be used. It takes \expr{String} as its only argument and returns \expr{Dynamic}, so it is advised to be assigned to a strictly typed variable.

\section{Flash}
\label{target-flash}

\section{Neko}
\label{target-neko}

\section{PHP}
\label{target-php}

\section{C++}
\label{target-cpp}

\subsection{Using C++ Defines}
\label{target-cpp-defines}
\begin{itemize}
    \item debug
    \item static_link
    \item HXCPP_M64
    \item haxe_210
    \item HXCPP_MULTI_THREADED
    \item HXCPP_FLOAT32
    \item HXCPP_CHECK_POINTER
    \item HXCPP_STACK_TRACE
    \item HXCPP_STACK_LINE
    \item HXCPP_STACK_VARS
    \item HXCPP_DEBUGGER
    \item HXCPP_GC_MOVING
    \item dll_import
    \item dll_import_include
    \item dll_export
\end{itemize}

\subsection{Using C++ Pointers}
\label{target-cpp-pointers}

\section{Java}
\label{target-java}

\section{C\#}
\label{target-cs}
